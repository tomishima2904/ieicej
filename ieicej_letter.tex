%% v3.3N [2023/03/30]
% \documentclass[paper]{ieicej}%「論文」形式になります
\documentclass[letter]{ieicej}%「レター」形式になります
% \documentclass[electronicsletter]{ieicej}%「レター(C分冊)」形式
%% 「技術研究報告」形式は tecrep.tex をコンパイルしてください
%\usepackage[dvipdfmx]{graphicx}
%\usepackage[dvips]{graphicx}
\usepackage{graphicx}
\usepackage[fleqn]{amsmath}
\usepackage{amsthm}
\usepackage{newtxtext}
\usepackage[varg]{newtxmath}

\def\IEICEJcls{\texttt{ieicej.cls}}
\def\IEICEJver{3.3}
\newcommand{\AmSLaTeX}{%
 $\mathcal A$\lower.4ex\hbox{$\!\mathcal M\!$}$\mathcal S$-\LaTeX}
\newcommand{\PS}{{\scshape Post\-Script}}
\def\BibTeX{{\rmfamily B\kern-.05em{\scshape i\kern-.025em b}\kern-.08em
 T\kern-.1667em\lower.7ex\hbox{E}\kern-.125em X}}

\newtheorem{theorem}{定理}% [section]

\field{A}
%\typeofletter{研究速報}
%\typeofletter{紙上討論}
%\typeofletter{問題提起}
%\typeofletter{ショートノート}
%\typeofletter{訂正}
\jtitle[電子情報通信学会論文誌 p\LaTeXe\ クラスファイルの使い方]%
       {電子情報通信学会論文誌 p\LaTeXe\ クラスファイル\\
        (ieicej.cls version \IEICEJver)の使い方}
\etitle{How to Use p\LaTeXe\ Class File (ieicej.cls version \IEICEJver) 
        for the Transactions of the Institute of Electronics, Information 
        and Communication Engineers}
\makeatletter
\if@paper
\makeatother
 \authorlist{%
  \authorentry[tomisima2904@chiba-u.jp]{富島 諒}{Ryo TOMISHIMA}{Chiba}
   \MembershipNumber{1111111}
  \authorentry[hory@faculty.chiba-u.jp]{堀内 靖雄}{Yasuo HORIUCHI}{Chiba}\MembershipNumber{2222222}
  \authorentry[kuroiwa@faculty.chiba-u.jp]{黒岩 眞吾}{Shingo KUROIWA}{Chiba}\MembershipNumber{}
 }
\else
 \authorlist{%
  \authorentry[tomisima2904@chiba-u.jp]{富島 諒}{Ryo TOMISHIMA}{m}{Chiba}
   \MembershipNumber{1111111}
  \authorentry{堀内 靖雄}{Yasuo HORIUCHI}{m}{Chiba}\MembershipNumber{2222222}
  \authorentry{黒岩 眞吾}{Shingo KUROIWA}{m}{Chiba}\MembershipNumber{}
 }
\fi
\affiliate[Chiba]{千葉大学,千葉県}
 {Department of Applied and Cognitive Informatics
 Division of Mathematics and Informatics
 Graduate School of Science and Informations, Chiba University, 
  1--33 Yayoicho,  Inage-ku, Chiba-shi, Chiba, 
  263--8522 Japan}
\jalcdoi{???????????}
%\affiliate[Tokyo]{%
% \Jorganization{第一大学}
% \Jdepartment{工学部},
% \Jaddress{東京都}
%}
%{\Edepartment{Faculty of Engineering},
% \Eorganization{First University},
% \Eaddress{1--2--3 Yamada, Minato-ku, Tokyo, 105--0123 Japan}
%}
%\affiliate[Osaka]{%
% \Jorganization{大阪株式会社},
% \Jdepartment{開発部},
% \Jaddress{吹田市}
%}
%{\Edepartment{R\&D Division},
% \Eorganization{Osaka Corporation},
% \Eaddress{4--5--6 Kawada, Suita-shi, 565--0456 Japan}
%}

\begin{document}
\makeatletter
\if@letter
\makeatother
\maketitle
\fi
\begin{abstract}
電子情報通信学会論文誌の p\LaTeXe\ クラスファイル,
\IEICEJcls{}(\texttt{version \IEICEJver})の使い方を説明します.
本クラスファイルに基づく記述の仕方,クラスファイル使用上の注意点,
ならびにタイピングの際の注意事項です.
本クラスファイルは,アスキー版 p\LaTeXe\ に基づいて作成しています.
\end{abstract}
\begin{keyword}
アスキー版p\LaTeXe{},タイピングの注意事項
\end{keyword}
\begin{eabstract}
IEICE (The Institute of Electronics, Information and Communication Engineers) 
provides a p\LaTeXe\ class file, named \IEICEJcls\ (ver.\,\IEICEJver), 
for the IEICE Transactions. This document describes how to use 
the class file, and also makes some remarks about typesetting 
a manuscript by using the p\LaTeXe. 
The design is based on ASCII Japanese p\LaTeXe. 
\end{eabstract}
\begin{ekeyword}
p\LaTeXe\ class file, typesetting, math formulas
\end{ekeyword}
\makeatletter
\if@letter
\makeatother
\else
 \maketitle
\fi

\section{まえがき}

語連想タスクとは,ある単語を提示し,その刺激語から連想される単語を即座に答えるタスクであり,
自然言語処理を含め幅広い分野で研究に利用されている\cite{liu-etal-2022-wax,de2019small}.

本論文では,事前学習済み言語モデル(Pre-trained Language Models; PLMs)を用いて,
与えられた刺激語に対し人が連想した連想語の,連想理由を推定・説明する手法の検討を行う.
現在,日本語における人間の語連想理由の説明を行うための大規模なデータセットが存在しないため,
本論文では in-context learning と呼ばれる,モデルパラメータの追加や調整を行わず,
モデルへの入力(プロンプト)に解かせたいタスクの入出力例(サンプル)を複数含める手法(few-shot)を用いた.
この際,プロンプトに使用するサンプルの質が下流タスクのパフォーマンスに大きく影響を与えることが知られている.
そこで,本論文では PLMs で in-context learning により,人の語連想理由の説明を行わせるにあたり,
プロンプトに使用するサンプルの自動生成手法と,生成されたサンプルから有効なサンプルを選択する手法を提案する.


\section{提案手法}
\label{sec:proposed_method}


\subsection{Zero-shotで連想理由候補生成}
まず,「与えられた入力語から出力語が連想される理由について解答しなさい\textbackslash n 入力語: イルカ 出力語: 水族館\textbackslash n 連想理由: 」
のように,本論文の著者が手動で作成した命令と連想語頻度表\cite{水野りか2011連想語頻度表}にある
刺激語(入力語)と連想語(出力語)を含むプロンプトを PLM(GPT)へ入力し,zero-shot で連想理由を30個生成する.


\subsection{連想理由候補のスコアリング}
次に,この生成した連想理由候補30個の中から few-shot 連想理由生成のサンプルとして使用する連想理由を選択する.
はじめに, PLM が補完した連想理由候補の選択の前処理として,
生成された連想理由内に刺激語と連想語のいずれかが含まれていないものは除外する.
続いて,生成された連想理由の刺激語と連想語のトークンを各々MASK トークンに置換した2文を用意する.
次に,用意した2文を RoBERTa\cite{liu2019roberta}に入力し,
Masked Language Model を使用して MASK トークン置換前のトークンを予測する.
そして, 置換前のトークンの予測順位を利用したスコア関数である,式\eqref{eq:scoring_1}によるスコアを第一基準に,よりスコアの高い連想理由を選択する.
\begin{align}
\label{eq:scoring_1}
    score &= \frac{1}{2}・\left( \frac{1}{rank_{\rm{cue}}} + \frac{1}{rank_{\rm{res}}} \right)
\end{align}
ここで$rank_{\rm{cue}}$は刺激語を MASK トークンに置換した際の予測順位,$rank_{\rm{res}}$は連想語を MASK トークンに置換した際の予測順位を表す.
この際,式\eqref{eq:scoring_1}のスコア関数では 同値となる連想理由が現れるため,文長を第二の基準としてより短いものを選択する.
さらに同値の場合は,RoBERTa による予測単語の softmax の値の和を第三の基準としてより大きいものを選択する.
また,予測語群中に正解となる刺激語または,連想語が含まれていない場合,$1/rank_{\rm{cue}}$および$1/rank_{\rm{res}}$は0とする.


\subsection{サンプルに使用する刺激語と連想語の組の選択}
さらに,本論文では25組の刺激語と連想語の組の中から,サンプルとして5組を選択するため,刺激語と連想語の組間での連想理由の比較も行う必要がある.
そこで,ある刺激語と連想語の組に対し,生成された連想理由候補の$score$の上位$k$個の平均値が大きい5組を few-shot プロンプトのサンプルとして含める連想理由として採用する.


\section{実験}
\label{sec:experiment}
Few-shot での語連想理由推定において,提案手法によるサンプルの自動生成・選択が有効であるかどうか調べるための実験を行う.
使用したモデルは東京大学松尾研究室の Weblab-10B\footnote{https://huggingface.co/matsuo-lab/weblab-10b} と
Rinna の日本語版 RoBERTa\footnote{https://huggingface.co/rinna/japanese-roberta-base} である.


\subsection{データセット}
\label{sec:dataset}
人間の語連想理由を説明させる際に,モデルに入力するための日本語の刺激語と連想語の組を用意する必要がある.
そこで,テスト用データセットと訓練用データセットの両者ともに,連想語頻度表\cite{水野りか2011連想語頻度表}から, 使用する刺激語と連想語の組を選択した.
訓練用データセット,テスト用データセットでそれぞれ 25 組,50組の刺激語と連想語の組を選定した.


\subsection{比較手法}
\label{sec:comparisons}
4条件で連想理由サンプルの生成を行った.
\begin{enumerate}
    \item {\bf Zero-shot}: テスト用データセットに対し, 連想理由の説明をサンプルを提示せずに zero-shot で連想理由を生成
    \item {\bf 自動ランダム}: zero-shotで生成したfew-shot 連想理由生成用のサンプルのうち,補完された連想理由内に刺激語と連想語の両方を含む理由の中からランダムに few-shot 連想理由生成用のサンプルを選択し,テスト用データセットに対し few-shot で連想理由を生成
    \item { \bf 提案手法}: 提案手法によって few-shot 連想理 由生成用のサンプルを生成・選択し,テスト用データセットに対し few-shot で連想理由を生成
    \item {\bf 手動1,手動2,手動3}: 20代男子大学生3名が手動で連想理由を作成したものを,それぞれ few-shot 連想理由生成用サンプルとして使用し,テスト用データセットに対し few-shot で連想理由を生成
\end{enumerate}


\subsection{評価方法}
\label{sec:evaluation}
\ref{sec:comparisons}節の条件のもと生成した連想理由に対し,20 代男子大学生10名が,人手による主観評価を行った.
主観二値評価では,上記の条件1つにつき3つの連想理由を生成し,それらをランダムに並び替えて被験者に提示した(一人あたり 900 文=50 組×3 文×6 条件).
被験者は提示された連想理由に対し,刺激語から連想語が連想される理由の説明に相応し いものならば 1,そうでないならば 0 の二値評価を行う.


\subsection{実験結果}

\begin{table}[tb]% Table 1
    \caption{Few-shotによる連想理由の説明を行った結果.各スコアの範囲は[0,3]}
    \label{table:results-few-shot}
    \begin{center}
    \begin{tabular}{ll}
    \Hline
    \noalign{\vskip.5mm}
    条件   & スコア \\
    \hline
    Zero-shot & 0.53 \\
    自動ランダム & 0.96 \\
    提案手法 & 1.4 \\
    手動1 & 1.5 \\
    手動2 & 1.8 \\
    手動3 & 1.7 \\
    \noalign{\vskip.5mm}
    \Hline
    \end{tabular}%
    \end{center}
    \end{table}

\label{sec:results}
図\ref{table:results-few-shot}に実験結果を示す.表のスコアは,各連想語と刺激語の組に対して出力された連想理由3文中に,
連想の理由の説明に相応しいものをいくつ含んでいたかの平均値を表す
(= 相応しいと判断した文の数/50 組/10 人).

まず,zero-shot に対して,ランダムで選択したものを含め,few-shot にすることでスコアが向上しており,
few-shot 提示による in-context learning の有効性が確認された.
次に,提案手法が自動ランダムに対して高いスコアを得ていることから,提案手法による MLM を利用した連想理由の選択方法が有効であることがわかる.
また,提案手法のスコアが, 3つの手動のスコアにかなり近づけたことから,事前学習済み大規模言語モデルを用いた連想理由の説明を生成する上で,
提案手法が有効な few-shot サンプルの自動生成・選択手法であると結論できる.


\section{まとめ}
本論文では,PLMs を用いて,与えられた刺激語に対し人が連想した連想語の,連想理由を推定・説明する手法を提案した.
連想理由説明文生成実験では,人手によって生成されたサンプルを使用した場合に近いスコアを示した.
今後の課題として,マルチホップ語連想理由タスクへの取り組みが考えられる.


\bibliography{bibs} %hoge.bibから拡張子を外した名前
\bibliographystyle{sieicej.bst} %参考文献出力スタイル


\makeatletter
\if@letter\else
\if@electronicsletter\else
\makeatother
\begin{biography}
\profile{m}{電子 花子}{%
1996東北一大学情報工学科卒.
1999東京第一大学工学部工学部助手.
某システムの研究に従事.
}
\profile{m}{情報 太郎}{%
1995大阪一大学工学科卒.
1997同大大学院工学研究科修士課程了.
1998大阪(株)入社.
某コンピュータ応用の研究に従事.
ABC学会会員.
}
\profile{n}{通信 次郎}{%
1980九州一大学理工学部卒.
1981大阪(株)入社.
現在ATT日本研究所に所属.
}
\end{biography}
\fi\fi

\end{document}